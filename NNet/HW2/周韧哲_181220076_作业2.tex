%!TEX program = xelatex
\documentclass[a4paper,UTF8]{article}
\usepackage{ctex}
\usepackage[margin=1.25in]{geometry}
\usepackage{color}
\usepackage{graphicx}
\usepackage{amssymb}
\usepackage{amsmath}
\usepackage{amsthm}
\usepackage{enumerate}
\usepackage{bm}
\usepackage{hyperref}
\usepackage{epsfig}
\usepackage{color}
\usepackage{mdframed}
\usepackage{lipsum}
\usepackage{graphicx}
\usepackage{float}
\newmdtheoremenv{thm-box}{Theorem}
\newmdtheoremenv{prop-box}{Proposition}
\newmdtheoremenv{def-box}{定义}

\usepackage{listings}
\usepackage{xcolor}
\lstset{
	numbers=left,
	numberstyle= \tiny,
	keywordstyle= \color{ blue!70},
	commentstyle= \color{red!50!green!50!blue!50},
	frame=shadowbox, % 阴影效果
	rulesepcolor= \color{ red!20!green!20!blue!20} ,
	escapeinside=``, % 英文分号中可写入中文
	xleftmargin=2em,xrightmargin=2em, aboveskip=1em,
	framexleftmargin=2em
}

\usepackage{booktabs}

\setlength{\evensidemargin}{.25in}
\setlength{\textwidth}{6in}
\setlength{\topmargin}{-0.5in}
\setlength{\topmargin}{-0.5in}
% \setlength{\textheight}{9.5in}
%%%%%%%%%%%%%%%%%%此处用于设置页眉页脚%%%%%%%%%%%%%%%%%%
\usepackage{fancyhdr}
\usepackage{lastpage}
\usepackage{layout}
\footskip = 12pt
\pagestyle{fancy}                    % 设置页眉
\lhead{2020年秋季}
\chead{神经网络}
% \rhead{第\thepage/\pageref{LastPage}页}
\rhead{作业一}
\cfoot{\thepage}
\renewcommand{\headrulewidth}{1pt}  			%页眉线宽,设为0可以去页眉线
\setlength{\skip\footins}{0.5cm}    			%脚注与正文的距离
\renewcommand{\footrulewidth}{0pt}  			%页脚线宽,设为0可以去页脚线

\makeatletter 									%设置双线页眉
\def\headrule{{\if@fancyplain\let\headrulewidth\plainheadrulewidth\fi%
\hrule\@height 1.0pt \@width\headwidth\vskip1pt	%上面线为1pt粗
\hrule\@height 0.5pt\@width\headwidth  			%下面0.5pt粗
\vskip-2\headrulewidth\vskip-1pt}      			%两条线的距离1pt
 \vspace{6mm}}     								%双线与下面正文之间的垂直间距
\makeatother

%%%%%%%%%%%%%%%%%%%%%%%%%%%%%%%%%%%%%%%%%%%%%%
\numberwithin{equation}{section}
%\usepackage[thmmarks, amsmath, thref]{ntheorem}
\newtheorem{theorem}{Theorem}
\newtheorem*{definition}{Definition}
\newtheorem*{solution}{Solution}
\newtheorem*{prove}{Proof}
\newcommand{\indep}{\rotatebox[origin=c]{90}{$\models$}}

\usepackage{multirow}

%--

%--
\begin{document}
\title{神经网络\\
作业二}
\author{181220076, 周韧哲, zhourz@smail.nju.edu.cn}
\maketitle

\section*{Problem 1}
设计一个多输入单输出的神经元用于进行股票价格的预测,通过股票的历史数据来训练该神经元,希望预测结果尽量接近真实值。
\begin{solution}.\\
	设股票价格与前$n$天有关,则神经元可以设计为$$f(\mathbf{x})=w_1x_1+w_2x_2+\cdots+w_nx_n+b=\mathbf{w}^T\mathbf{x}+b$$
	令$\mathbf{X}=(\mathbf{x};1),\mathbf{W}=(\mathbf{w};b)$,则神经元可表示为$f(\mathbf{X})=\mathbf{W}^T\mathbf{X}$。按照课件$101$页所示,令$n=4$,使用所给数据训练该网络,学习率$\alpha=0.001$,误差使用均方误差,迭代次数为$1000$次,得到训练后的均方误差$0.9847$,权重$[w_1,w_2,w_3,w_4,b]$为
	$$[-14.17614168,\quad-4.345092,\quad4.46244507,\quad14.29241046,\quad9.67450308]$$
	详见代码文件。
\end{solution}

\section*{Problem 2}
“损坏的”LED灯问题。考虑一个由7个LED灯组成的数字显示器,每个LED灯的亮暗状态分别标记
为“+1”和“−1”,这7个LED灯的状态共同组成一个向量$x$。显示器上显示的数字标记为s。例如当s=2时,第j个LED灯(
j=1,...,7)显示为$c_j(2)$(即正确显示)的概率为f,或者翻转显示(即错误显示)的概率为f。假定显示的数字只为2或3,当给定一个显示状态x时,显示数字为2或3的概率为多少?例如$P(s=2|x)$可以写成如下形式$P(s=2|x)=\frac{1}{1+exp(-w^Tx+b)}$。这里f=0.1。
\begin{solution}.\\
	已知$P(x|s)=(1-f)^7$,为了求$P(s|x)$,可以使用贝叶斯定理
	$$P(s|x)=\frac{P(x|s)P(s)}{P(x)}$$
	我们并不知道$P(s)$,无法求解。但是,我们可以观察每一次事件,如果将每种概率事件作为训练数据来训练一个神经网络,从而预测显示数字s为2或者3的概率,即$$P(s=2|x)=\frac{1}{1+exp(-w^Tx+b)}$$
	训练完后我们就可以得到参数$w,b$,从而可以得到$P(s=2|x)$与$P(s=3|x)$。
\end{solution}
\end{document}